\documentclass[12pt]{article}


%-------Pakete-------%

%Rechtschreibung & Layout
\usepackage[utf8]{inputenc}
\usepackage[T1]{fontenc}
\usepackage[left=2cm,right=2cm,top=2cm,bottom=2cm]{geometry}
\usepackage{lmodern}
\usepackage[english]{babel}
\usepackage{booktabs,multirow}
\usepackage{csquotes}
\usepackage{icomma}
\usepackage[singlespacing]{setspace}
\usepackage{caption}
\usepackage{subcaption}

%Pakete: Grafiken & Anpassung
\usepackage{graphicx}
	\graphicspath{{Grafiken/}}
\usepackage{wrapfig}
\usepackage{wallpaper}
\usepackage{tikz}
\usepackage{witharrows}
\usepackage{tcolorbox}
\tcbuselibrary{most}
\usepackage{lipsum}

\newtcolorbox{aufgabe}{breakable,colframe=cyan,colback=cyan!20}

%Pakete: Inhaltsverzeichnis & Referenzierung
\usepackage[natbib,abbreviate=true,doi=false,style=numeric-comp,giveninits=true,sorting=none]{biblatex}
	\addbibresource{MyBibliography.bib}
\usepackage{color}
\usepackage{hyperref}
%\usepackage[all]{hypcap}
\hypersetup
{
    colorlinks=false, % make the links colored
    linkcolor=blue, % color TOC links in blue
    urlcolor=red, % color URLs in red
    linktoc=section % 'all' will create links for everything in the TOC
}

%Pakete: Mathematik
\usepackage{amssymb}
\usepackage{amsmath}
\usepackage{amsfonts}
\usepackage{amsthm}
\usepackage[locale = DE,space-before-unit=true,per-mode = symbol]{siunitx}
\usepackage{nicematrix}
\NiceMatrixOptions{
code-for-first-row = \color{blue!50} ,
code-for-first-col = \color{blue!50} ,
}

\newtheorem{definition}{Definition}[subsection]
\newtheorem{example}{Beispiel}[subsection]
\newtheorem{theorem}{Theorem}[subsection]
\newtheorem{lemma}{Lemma}[subsection]

\setcounter{biburllcpenalty}{9000}% Kleinbuchstaben
\setcounter{biburlucpenalty}{9000}% Großbuchstaben

\begin{document}

\section{Introductory Example}

Explanation of the phenomenon and warm up.

\section{The General Lotka-Volterra Equations}
\subsection{Simple Predator-Prey Model $(n = 2)$}

\begin{itemize}
    \item Derivation of the 2-dimensional predator-prey model as a special case of the general system of equations
    \item Implications for the environment and ecosystem
    \item Numerical simulations and theoretical results regarding the predator-prey model, including stability analysis, slope field, phase space
    \item predator-pest model as a consequence from the independence of the population mean from the initial value. Application time is crucial to decrease the size of the pest successfully.
    \item Methods: Python (FORTRAN)
\end{itemize}

\subsection{Competitive System}
\begin{itemize}
    \item Derivation of the Lotka-Volterra competition model as another special case of the general system of equations
    \item Implications for the environment and ecosystem (logistic growth)
    \item Numerical classification of the four cases in two dimensions; stability analysis, slope field, phase space
    \item Competitive exclusion principle as a consequence of the classification of the four different cases
    \item Methods: Python (FORTRAN)
\end{itemize}

\section{Extended Models}

Short summary of the problems of the general system and ways to refine the equations by adding more terms for different ecological effects, such as:
\begin{itemize}
    \item Seasons
    \item Interaction rates changing with respect to the past (delay kernels)
\end{itemize}



\newpage
\nocite{*}
\printbibliography
\end{document}